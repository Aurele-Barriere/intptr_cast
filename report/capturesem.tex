\label{sec:capturesem}

\begin{figure}
\begin{subfigure}{.48\textwidth}
  \begin{lstlisting}[language=C]
int main() {
  int a = 0;
  int b = (int) &a;
  return 0;
}
  \end{lstlisting}
  \caption{C source program}
  \label{fig:capturec}
\end{subfigure}
\begin{subfigure}{.48\textwidth}
  \begin{lstlisting}[language=C]
int main(void)
{
  int a;
  int b;
  a = 0;
  builtin builtin __capture(&a);
  b = (int) &a;
  return 0;
  return 0;
}
  \end{lstlisting}
  \caption{Clight program}
  \label{fig:captureclight}
\end{subfigure}
\caption{CompCert now automatically inserts the capture function before casts}
\label{fig:captureexample}
\end{figure}



In this section, the implementation has been done by Juneyoung Lee.

The capture function is the new builtin function to turn logical blocks into concrete blocks.
To be sure that casting from pointer to integer is always possible, the capture function should be called before every such cast.

The function is added in the first verified pass of CompCert, when translating CompCert C programs into Clight.

The results can be seen on the Figure~\ref{fig:captureexample}. We can notice the capture of the memory block of \texttt{a} just before casting it's address to an integer.

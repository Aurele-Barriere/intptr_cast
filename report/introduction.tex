When compiling critical software written in C, one expects from the compiler to not introduce any bugs, or any behavior that wasn't specified in the C source code.

% compcert and coq
To meet this expectation, CompCert~\cite{DBLP:journals/cacm/Leroy09} is a formally verified C compiler.
It uses the Coq Proof Assistant~\cite{coq} to prove that the compiled code and the source code have the same observable behavior, as defined by the ISO C Standard~\cite{iso}.
CompCert also aims to provide performance of the generated code, and implements several common optimizations. Compiled code runs approximately 10\% slower than code compiled with \texttt{GCC4 -O1}~\cite{compcertwebsite}.
As of today, CompCert is a trusted compiler; despite many efforts~\cite{DBLP:conf/pldi/YangCER11}, no bug have been discovered within the verified parts of CompCert.
CompCert currently supports all of the ISO C 99 Standard, with very few exceptions~\cite{compcertwebsite}.

% iso c and semantics
However, the ISO C standard itself does not define semantics for every syntactically valid C program.
Many C programs are said to have \textit{unspecified behavior} or \textit{undefined behavior}, meaning that conforming compilers can produce any compiled code.
Despite the lack of semantics, many C programmers are using such programs and expect a precise result. %not a very good sentence
This leads to difficult bugs~\cite{DBLP:conf/apsys/WangCCJZK12} and the impossibility of proving that the compiled code behaves as expected.

% integer pointer casts
One popular unspecified feature of the C language is the casting between integer and pointer values.
Such casts have many uses in real C programs. \todo{examples: Linux Kernel, JVM, others could be found in (http://www.cl.cam.ac.uk/~pes20/cerberus/notes50-survey-discussion.html).}
When compiled with most compilers, those programs behave as expected from the programmers. But these intuitive semantics have not been formalized in the C standard yet.

% motivation of the Kanget al. paper
Defining a precise, formal semantics for integer-pointer casting and pointer manipulation would allow CompCert to compile even more C programs in a certified way.
The semantics of pointer manipulation depends on the memory model of the compiler.
As of today, CompCert uses a logical memory model~\cite{leroy:hal-00703441}, where every memory block is an abstract object without a concrete memory address. Such a memory model enables many optimizations, because a program can never guess the location of a block and modify it without a pointer. \todo{example}.
However, integer-pointer casting isn't possible.
Other works have investigated using a concrete memory model, to reflect the state of a real machine's memory \addref{ref}. But then, most optimizations cannot be done anymore without changing the behavior of the program \todo{either go back to the example or reference a later section}.
In~\cite{DBLP:conf/pldi/KangHMGZV15}, Kang et al suggest a quasi-concrete model, in which there are both logical and concrete memory blocks. The main idea is to use logical blocks by default, that can allow optimizations, and use concrete blocks when the concrete address of a memory block is needed.

% contribution
This new memory model has been implemented in CompCert.
In this paper, we discuss this implementation.
We show that it is relevant and supports integer-pointer casts while still allowing common optimizations.
We present the difficulties of the implementation, and the changes that needed to be done in CompCert.
% this should be bigger

% outline
At first, we remind the reader about the different memory models.
Then we present the implementation of the new memory model. The next part shows how the notion of memory injections had to be changed. Then, we present our mixed simulation reasoning, used to prove the correctness of the extended language. After that, we present the semantics of integer-pointer casting and of the \texttt{capture} function.
Finally, we discuss the results of the implementation and its effect on optimizations. 
% related works?

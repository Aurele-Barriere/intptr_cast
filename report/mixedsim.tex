\label{sec:mixedsim}
In the quasi-concrete memory model, with the addition of the \textit{capture} function, every language of CompCert is now non-deterministic. Indeed, giving a concrete address to a logical block is a non-deterministic operation, whereas the logical allocation used so far in CompCert was deterministic.

\subsection{Simulations in CompCert and properties}
To prove the correctness of CompCert, one must prove that the behavior of the generated target code is one of the behaviors of the original source code. 
To do so, CompCert uses \textit{backward simulations} between the semantics of the source code and the generated code.
In CompCert, the semantics of each language is defined. The semantics of a program is a set of states (with initial and final states) and a relation between states meaning that one can go from one state to another whith a given observable trace.
\todo{pictures? see compcert\_overview.pdf}

\subsubsection{Definition: Semantics of a program p}: $\mathit{Sem}(p) = (\states{p},I_p,F_p,\step{p})$ where $\step{p}\subseteq\states {p}\times\traces\times\states{p}$ and $I,F\subseteq\states{p}$. $t\in\traces$ is a trace, \textit{i.e.} a list of observable events (system call, store or load).

A straightforward way to show that two programs $p_s$ and $p_t$ are semantically equivalent could be to show that their semantics are \textit{bisimulated}, meaning that there exists a relation between states of the source and target semantics such that every time the source program takes a step to a new state, the target program takes a step with the same trace to a matching state, and vice versa. However, such a relation is in general too strong, as it does not allow basic optimizations.

A backward simulation is a weaker simulation that proves semantics preservation while still allowing optimizations.

\subsubsection{Definition: Backward Simulation} Let $sp$ a source program and $tp$ a target program. A backward simulation is a relation $R$ such that
\begin{itemize}
\item $\forall i\in I_{tp}, \exists i'\in I_{sp}, (i,i')\in R$
\item $\forall t\in\traces, ss_1\in\states{sp},st_1,st_2\in\states{tp},$\\
      $\quad(ss_1,st_1)\in R~\wedge~(st_1,t,st_2)\in\step{tp}\implies$\\
      $\exists ss_2\in\states{sp}, (ss_1,t,ss_2)\in\step{sp}$
\end{itemize}
Informally, each time the target program takes a step to a new state, the source program also takes a step to a matching state. This is enough to claim that the target's behavior is a refinment of the source's behavior.
\todo{some tree picture to describe backward sim}

\subsubsection{Forward Simulations} One can similarly define a forward simulation in a symmetric way: each time the source program takes a step to a newt state, the target program also takes a step to a matchin state. This means that the target's behavior extends the source's behavior.

\subsubsection{Theorem: forward to backward} If the target language is \textit{determinate} and the source language is \textit{receptive} and there is a forward simulation between the semantics of $sp$ and $tp$, then we can construct a backward simulation between the semantics of $sp$ and $tp$. \todo{definition of receptive and determinate}

\subsubsection{Theorems: simulation composition} Let $S_1,S_2,S_3$ be semantics.
If there is are forward simulations between $S_1$ and $S_2$, and between $S_2$ and $S_3$, then there is a forward simulation between $S_1$ and $S_3$.
If there is are backward simulations between $S_1$ and $S_2$, and between $S_2$ and $S_3$, then there is a backward simulation between $S_1$ and $S_3$.


\subsection{Atomic Semantics and properties}
\subsubsection{Definition: Atomic Semantics} $\mathit{Atomic}(\states{p},I_p,F_p,\step{p}) = (\traces\times\states{p},\{[]\}\times I_p,\{[]\}\times F_p,\atstep{p})$ where $[]$ is the empty trace and $\atstep{p}$ is defined as follow:
\begin{itemize}
\item $(s_1,[],s_2)\in\step{p}\implies(([],s_1),[],([],s_2))\in\atstep{p}$
\item $(s_1,ev::t,s_2)\in\step{p}\implies(([],s_1),[ev],(t,s_2))\in\atstep{p}$
\item $((ev::t,s),[ev],(t,s))\in\atstep{p}$
\end{itemize}
Informally, $\mathit{Atomic}$ of a semantics is the same semantics where steps with multiple events have been replaced with several single-event steps. This is illustrated figure \todo{example}.

\subsubsection{Theorem: Factor forward simulation} If there is a forward simulation between $S_1$ and $S_2$ and $S_2$ has \textit{single events}, then there is a forward simulation between $\mathit{Atomic}(S_1)$ and $S_2$.

\subsubsection{Theorem: Factor backward simulation} If there is a backward simulation between $S_1$ and $S_2$ and $S_1$ has \textit{single events}, then there is a forward simulation between $S_1$ and $\mathit{Atomic}(S_2)$.

\subsection{The correctness proof in CompCert}
To prove the correctness of the compilation, we need to prove that for every source C program $sp$, if $tp$ is the compiled ASM program, then we have a backward simulation between $\mathit{Sem}(sp)$ and $\mathit{Sem}(tp)$.


\subsection{Non-determinism}
\subsection{Mixed simulations and properties}
\subsection{The new correctness proof}

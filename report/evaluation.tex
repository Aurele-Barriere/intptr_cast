\label{sec:eval}
\begin{figure}[H]
\begin{subfigure}{.48\textwidth}
  \begin{lstlisting}[language=C,basicstyle=\small]
extern void g();
int f() {
  int a = 0;
  int* q = &a;
  g();
  return a; }
\end{lstlisting}
\end{subfigure}
\begin{subfigure}{.48\textwidth}
  \begin{lstlisting}[basicstyle=\small]
f() {
    6:	x4 = 0
    5:	int32[stack(0)] = x4
    4:	nop
    3:	x3 = "g"()
    2:	x2 = 0
    1:	return x2 }
\end{lstlisting}
\end{subfigure}
\caption{Constant propagation on logical blocks}
\label{fig:cplogical}
\end{figure}
\begin{figure}[H]
\begin{subfigure}{.48\textwidth}
  \begin{lstlisting}[language=C,basicstyle=\small]
extern void g();
int f() {
  int a = 0;
  int* q = &a;
  int b = (int) q;
  g();
  return a; }
\end{lstlisting}
\end{subfigure}
\begin{subfigure}{.48\textwidth}
  \begin{lstlisting}[basicstyle=\small]
f() {
    8:	x5 = 0
    7:	int32[stack(0)] = x5
    6:	x2 = stack(0) (int)
    5:	_ = builtin __capture(x2)
    4:	nop
    3:	x4 = "g"()
    2:	x3 = int32[stack(0)]
    1:	return x3 }
\end{lstlisting}
\end{subfigure}
\caption{No Constant propagation on concrete blocks}
\label{fig:cpconcrete}
\end{figure}

The correctness of CompCert with the new memory model hasn't been entirely proved yet.
The remaining proofs are most of the proofs of mixed simulations.
However, we remain confident that such a mixed simulation exists for every pass. One mixed simulation has been proved (\texttt{CSEproof}, an optimization pass), and others should be very similar. The intuition behind any mixed simulation proof is the same: use the previous forward simulation proof to prove a local forward simulation of non-external states, then use the new properties of external calls to prove the local backward simulation of external states.

Even if the semantics of casting has not been fully implemented yet, it is already possible to compile some C programs with our modified version of CompCert. We can see that the new model successfully gives semantics to integer-pointer casting, but also allow optimizations when using logical blocks.
This is illustrated Figure~\ref{fig:cplogical}. We can see that constant propagation is done, and the program returns 0. This is explained by the fact that the memory block of \texttt{a} is logical, and thus the function \texttt{f()} has ownership of this memory block. The compiler deduce that \texttt{a} cannot be accessed by the function \texttt{g()}, and thus performs constant propagation.
However, if the block is made concrete, then \texttt{f()} looses ownership of \texttt{a} and constant propagation should not be done. This is illustrated Figure~\ref{fig:cpconcrete}, where the memory block of \texttt{a} has to be captured due to the pointer-to-integer cast.

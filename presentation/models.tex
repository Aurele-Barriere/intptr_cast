\subsection{Logical Memory Model}
\begin{frame}{\subsecname}

  \begin{block}{Memory blocks}
    \begin{itemize}
    \item $\texttt{Block}=\{(v,n,c)~|~v\in\texttt{Bool},n\in\mathbb{N},c\in\texttt{Val}^{n}\}$
    \item Infinite memory.
    \item Block ownership.
    \end{itemize}
  \end{block}
  \vfill
  \begin{center}
    \begin{tabular}{l c r}
      \lstinputlisting[linewidth=3cm]{listings/logical.c} &
      $\xrightarrow[\text{Dead Alloc Elimination}]{\text{Constant Propagation}}$ &
      \lstinputlisting[linewidth=3cm]{listings/logicaloptim.c}
    \end{tabular}
  \end{center}
  \vfill
  \begin{alertblock}{No semantics for Integer-Pointer cast (unspecified)}
  \end{alertblock}
  
\end{frame}

\subsection{Concrete Memory Model}
\begin{frame}{\subsecname}

  \begin{block}{Memory blocks}
    $$\texttt{Block}=\{(p,n)~|~p\in\texttt{int32},n\in\texttt{int32}\}$$
  \end{block}
  \vfill
  \begin{block}{Memory Consistency}
    \begin{itemize}
    \item \textbf{No overflow:} $\quad\forall (p,n)\in\texttt{Allocated}, [p,p+n]\subseteq]0,2^{32}[$.
      \item \textbf{No overlap:} $\quad\forall (p_1,n_1), (p_2,n_2)\in\texttt{Allocated},$\\ $[p_1,p_1+n_1]\text{ and }[p_2,p_2+n_2]\text{ are disjoint.}$
    \end{itemize}
  \end{block}
  \vfill
  \begin{alertblock}{No optimizations}
    No ownership of memory blocks.
  \end{alertblock}

\end{frame}

\subsection{Quasi-Concrete Memory Model}
\begin{frame}{\subsecname}

  \begin{block}{Memory blocks}
    \begin{itemize}
    \item $\texttt{Block}=\{(v,p,n,c)~|~v\in\texttt{Bool},n\in\mathbb{N},c\in\texttt{Val}^{n},p\in\texttt{int32}\cup\{\undef\}\}$.
    \item Memory consistency (\textbf{No overflow}, \textbf{No overlap}) for concrete blocks.
    \end{itemize}
  \end{block}
  \vfill
  \begin{exampleblock}{Properties}
    \begin{itemize}
    \item Optimizations are possible with logical blocks.
    \item Integer-Pointer cast is posible with concrete blocks.
    \end{itemize}
  \end{exampleblock}
  
\end{frame}

\begin{frame}{The Capture Function}

  \begin{tabular}{l c r}
    \lstinputlisting[linewidth=4.3cm]{listings/capture.c} & $\longrightarrow$ &
    \lstinputlisting[linewidth=4.3cm]{listings/captured.c}
  \end{tabular}
  \vfill
  \begin{alertblock}{Non-determinism}
    A logical block can be captured at several different addresses.
  \end{alertblock}
  
\end{frame}

\begin{frame}{Examples}
  \begin{tabular}{l c r}
    \lstinputlisting[linewidth=3cm]{listings/logical.c} &
    $\xrightarrow[\text{Quasi-Concrete Model}]{\text{Logical Model}}$ & 
    \lstinputlisting[linewidth=3.5cm]{listings/logicaloptim.c}
  \end{tabular}
  \vfill
  \begin{tabular}{l c r}
    \lstinputlisting[linewidth=4cm]{listings/concrete.c} &
    $\xrightarrow{\text{Quasi-Concrete Model}}$ &
    \lstinputlisting[linewidth=4cm]{listings/concretecaptured.c}
  \end{tabular}

\end{frame}

\def\backward{
  \begin{figure}
    \centering
    \begin{tikzpicture}[%
        every node/.style={circle,minimum size=3pt,minimum height=3pt, inner sep=0pt},
        shorten >=2pt,
        node distance=1cm
      ]
      \node [draw] (0) [color=blue, fill] {};
      \node [draw] (1) [color=blue, fill, below left=of 0] {};
      \node [draw] (2) [color=blue, fill, below=of 0] {};
      \node [draw] (3) [color=blue, fill, below right=of 0] {};
      \node [draw] (4) [color=blue, fill, below left=of 2] {};
      \node [draw] (5) [color=blue, fill, below=of 2] {};
      \node [draw] (6) [color=blue, fill, below=of 3] {};
      \node [draw] (7) [color=red, fill, right=of 0, xshift=1.5cm] {};
      \node [draw] (8) [color=red, fill, below left=of 7] {};
      \node [draw] (9) [color=red, fill, below=of 7] {};
      \node [draw] (10) [color=red, fill, below=of 8] {};
      \node [draw] (11) [color=red, fill, below=of 9] {};
      \path [draw] (0) edge[->]  node {} (1)
      (0) edge[->]  node {} (2)
      (0) edge[->]  node {} (3)
      (2) edge[->]  node {} (4)
      (2) edge[->]  node {} (5)
      (3) edge[->]  node {} (6)
      (7) edge[->]  node {} (8)
      (7) edge[->]  node {} (9)
      (8) edge[->]  node {} (10)
      (9) edge[->]  node {} (11);
      \draw[dashed] (0) -- (7);
      \draw[dashed] (3) -- (8);
      \draw[dashed] (2) -- (9);
      \draw[dashed] (5) -- (11);
      \draw[dashed] (6) -- (10);
    \end{tikzpicture}
    \label{fig:backward}
  \end{figure}
}

\def\forward{
  \begin{figure}
    \centering
    \begin{tikzpicture}[%
        every node/.style={rectangle,minimum size=3pt,minimum height=3pt, inner sep=0pt},
        shorten >=2pt,
        node distance=1cm
      ]
      \node [draw] (0) [color=red, fill] {};
      \node [draw] (1) [color=red, fill, below right=of 0] {};
      \node [draw] (2) [color=red, fill, below=of 0] {};
      \node [draw] (3) [color=red, fill, below left=of 0] {};
      \node [draw] (4) [color=red, fill, below right=of 2] {};
      \node [draw] (5) [color=red, fill, below=of 2] {};
      \node [draw] (6) [color=red, fill, below=of 3] {};
      \node [draw] (7) [color=blue, fill, left=of 0, xshift=-1.5cm] {};
      \node [draw] (8) [color=blue, fill, below right=of 7] {};
      \node [draw] (9) [color=blue, fill, below=of 7] {};
      \node [draw] (10) [color=blue, fill, below=of 8] {};
      \node [draw] (11) [color=blue, fill, below=of 9] {};
      \path [draw] (0) edge[->]  node {} (1)
      (0) edge[->]  node {} (2)
      (0) edge[->]  node {} (3)
      (2) edge[->]  node {} (4)
      (2) edge[->]  node {} (5)
      (3) edge[->]  node {} (6)
      (7) edge[->]  node {} (8)
      (7) edge[->]  node {} (9)
      (8) edge[->]  node {} (10)
      (9) edge[->]  node {} (11);
      \draw[dashed] (0) -- (7);
      \draw[dashed] (3) -- (8);
      \draw[dashed] (2) -- (9);
      \draw[dashed] (5) -- (11);
      \draw[dashed] (6) -- (10);
    \end{tikzpicture}
    \label{fig:forward}
  \end{figure}
}

\def\mixed{
  \begin{figure}
    \centering
    \begin{tikzpicture}[%
        every node/.style={rectangle,minimum size=4pt,minimum height=4pt, inner sep=0pt},
        shorten >=2pt,
        node distance=0.6cm
      ]
      \node [draw] (0) [color=blue, fill] {};
      \node [draw] (1) [circle, color=blue, fill, below=of 0] {};
      \node [draw] (2) [color=blue, fill, below left=of 1] {};
      \node [draw] (3) [color=blue, fill, below=of 1] {};
      \node [draw] (4) [color=blue, fill, below=of 3] {};
      \node [draw] (5) [color=red, fill, right=of 0] {};
      \node [draw] (6) [circle, color=red, fill, below=of 5] {};
      \node [draw] (7) [circle, color=red, fill, below right=of 5] {};
      \node [draw] (8) [color=red, fill, below=of 6] {};
      \node [draw] (9) [color=red, fill, below=of 8] {};
      \node [draw] (10) [color=red, fill, below right=of 8] {};
      \path [draw] (0) edge[->]  node {} (1)
      (1) edge[->]  node {} (2)
      (1) edge[->]  node {} (3)
      (3) edge[->]  node {} (4)
      (5) edge[->]  node {} (6)
      (5) edge[->]  node {} (7)
      (6) edge[->]  node {} (8)
      (8) edge[->]  node {} (9)
      (8) edge[->]  node {} (10);
      \draw[dashed] (0) -- (5);
      \draw[dashed] (1) -- (6);
      \draw[dashed] (3) -- (8);
      \draw[dashed] (4) -- (9);
    \end{tikzpicture}
    \label{fig:mixed}
  \end{figure}
}

\subsection{Memory Update}
\begin{frame}{\subsecname}

  \begin{block}{Updating the definition of the memory}
  Add a map \texttt{mem\_concrete: block -> Option Z}.
  \end{block}
  \vfill
  \begin{block}{Abstract Analysis}
    The stack block can be accessible if it is concrete.
  \end{block}
    \vfill
    \begin{block}{Memory Injection}
    Concrete blocks should be preserved at the same address.
    \end{block}
    \vfill
    \begin{exampleblock}{Proofs}
      \begin{itemize}
      \item Every memory operation preserves memory consistency.
      \item Abstract analysis is sound.
      \item Memory injections are preserved.
      \end{itemize}
  \end{exampleblock}

  
\end{frame}

%\subsection{Memory Injection}
%\begin{frame}{\subsecname}
%
%  \begin{block}{Memory Injection}
%    A map from some blocks of the source memory to blocks of the target memory.
%  \end{block}
%  \vfill
%  \begin{block}{Changes}
%  Concrete blocks in the source should be preserved, at the same address, in the target.
%  \end{block}
%  \vfill
%  \begin{exampleblock}{Proofs}
%    Every memory operation preserves memory injection.
%  \end{exampleblock}
%  
%\end{frame}
%
%\subsection{Capture Insertion}
%\begin{frame}{\subsecname}
%
%  \begin{block}{Insert capture before each cast}
%  Done by Juneyoung Lee.\\
%  Between CompCert C and Clight.
%  \end{block}
%  \vfill
%  \begin{tabular}{l c r}        
%    \lstinputlisting[linewidth=4cm]{listings/captureinsert.c} &
%    $\longrightarrow$ & 
%    \lstinputlisting[linewidth=6.2cm]{listings/captureinsert.clight}
%  \end{tabular}
%
%\end{frame}
%
\subsection{Mixed Simulations}
\begin{frame}{Proving the correctness of CompCert}
  \begin{columns}[T] % align columns
    \begin{column}{.48\textwidth}
      \begin{block}{Backward Simulation}
        Step in target $\rightarrow$\\ Matching step in source\\
        {\color{blue}Source\hfill\color{red}Target}
        \backward
      \end{block}
    \end{column}%
    \hfill
    \begin{column}{.48\textwidth}
      \begin{block}{Forward Simulation}
        Step in source $\rightarrow$\\ Matching step in target\\
        {\color{blue}Source\hfill\color{red}Target}
        \forward
      \end{block}
    \end{column}%
  \end{columns}
\end{frame}

\begin{frame}{Changing the correctness proof}
  \begin{exampleblock}{Goal}
    A backward simulation between C and ASM.
  \end{exampleblock}
  \vfill
  \begin{alertblock}{Previous proof}
    Determinacy + forward simulation $\rightarrow$ backward simulation.\\
    No more determinacy!
  \end{alertblock}
  \vfill
  \begin{exampleblock}{Handling non-deterministic behavior}
    Non-deterministic behavior only for the capture function.\\
    For every other state, we still have local determinacy.
  \end{exampleblock}
\end{frame}

\begin{frame}{Mixed Simulations}
  \begin{block}{Mixed Simulation}
    Backward state ($\bigcirc$): Local backward simulation.\\
    Forward state ($\Box$): Local forward simulation.\\
    {\color{blue}Source\hfill\color{red}Target}
    \vspace{-0.8cm}
    \mixed
  \end{block}
  \vfill
  \begin{block}{Theorem}
    MixedSim (A,B) $\rightarrow$ BackwardSim ((atomic A),B).
  \end{block}
\end{frame}


\begin{frame}{The new correctness proof}
  \vspace{-0.4cm}
  \proofmixedall
  \vspace{-0.5cm}
  \begin{block}{Theorems}
    \begin{enumerate}
    \item Mixed simulations.
    \item MixedSim (A,B) $\rightarrow$ BackwardSim ((atomic A),B).
    \item BackwardSim(A,B) $\rightarrow$ BackwardSim(A,atomic(B)).
    \item Backward Simulation Composition.
    \end{enumerate}
  \end{block}
  
\end{frame}
